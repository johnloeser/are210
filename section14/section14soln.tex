\documentclass[12pt,english]{article}
\usepackage{geometry}
\usepackage{float}
\usepackage{caption}
\geometry{verbose,tmargin=3cm,bmargin=3cm,lmargin=3cm,rmargin=3cm}
\usepackage{amsmath}
\usepackage{amssymb}
\usepackage{amsthm}
\usepackage{verbatim}
\usepackage{adjustbox}
\usepackage{hyperref}
\usepackage{graphicx}
\usepackage{setspace}
\usepackage{changepage}
\usepackage{enumitem}
\setlist{nolistsep}
\onehalfspacing
\usepackage{babel}
\newcommand{\expec}{\ensuremath{\mathbb E}}
\newcommand{\T}{\ensuremath{\text{T}}}
\begin{document}
\begin{center}
{\Large{}Section 14: Confidence regions}
\par\end{center}{\Large \par}

\begin{center}
ARE 210
\par\end{center}

\begin{center}
December 5, 2017
\par\end{center}

\begin{enumerate}
	\item Prove the result on the duality of $(1 - \alpha)$100\% confidence regions and level-$\alpha$ tests.
	\vspace{1em}
	
	Let $\delta_{v}$ be a level-$\alpha$ test of the null $H_{v} : \nu(P) = v$. Define $C_{v} = \{ X : \delta_{v}(X) = 1 \}$ and $A_{v} = \mathcal{X} \setminus C_{v}$. Let $S(X) = \{ v \in \mathcal{N} : X \in A_{v} \}$. Then $P_{P}[\nu(P) \in S(X)] = P_{P}[X \in A_{\nu(P)}] \geq 1 - \alpha$.
	
	Note that when $\nu$ is invertible, we can reverse this logic.
	\vspace{1em}
\end{enumerate}

\end{document}